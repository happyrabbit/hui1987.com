\documentclass[12pt,]{krantz}
\usepackage{lmodern}
\usepackage{amssymb,amsmath}
\usepackage{ifxetex,ifluatex}
\usepackage{fixltx2e} % provides \textsubscript
\ifnum 0\ifxetex 1\fi\ifluatex 1\fi=0 % if pdftex
  \usepackage[T1]{fontenc}
  \usepackage[utf8]{inputenc}
\else % if luatex or xelatex
  \ifxetex
    \usepackage{mathspec}
  \else
    \usepackage{fontspec}
  \fi
  \defaultfontfeatures{Ligatures=TeX,Scale=MatchLowercase}
    \setmonofont[Mapping=tex-ansi,Scale=0.7]{Source Code Pro}
\fi
% use upquote if available, for straight quotes in verbatim environments
\IfFileExists{upquote.sty}{\usepackage{upquote}}{}
% use microtype if available
\IfFileExists{microtype.sty}{%
\usepackage[]{microtype}
\UseMicrotypeSet[protrusion]{basicmath} % disable protrusion for tt fonts
}{}
\PassOptionsToPackage{hyphens}{url} % url is loaded by hyperref
\usepackage[unicode=true]{hyperref}
\PassOptionsToPackage{usenames,dvipsnames}{color} % color is loaded by hyperref
\hypersetup{
            pdftitle={Reading Notes},
            pdfauthor={Hui Lin},
            colorlinks=true,
            linkcolor=Maroon,
            citecolor=Blue,
            urlcolor=Blue,
            breaklinks=true}
\urlstyle{same}  % don't use monospace font for urls
\usepackage{natbib}
\bibliographystyle{apalike}
\usepackage{longtable,booktabs}
% Fix footnotes in tables (requires footnote package)
\IfFileExists{footnote.sty}{\usepackage{footnote}\makesavenoteenv{long table}}{}
\IfFileExists{parskip.sty}{%
\usepackage{parskip}
}{% else
\setlength{\parindent}{0pt}
\setlength{\parskip}{6pt plus 2pt minus 1pt}
}
\setlength{\emergencystretch}{3em}  % prevent overfull lines
\providecommand{\tightlist}{%
  \setlength{\itemsep}{0pt}\setlength{\parskip}{0pt}}
\setcounter{secnumdepth}{5}
% Redefines (sub)paragraphs to behave more like sections
\ifx\paragraph\undefined\else
\let\oldparagraph\paragraph
\renewcommand{\paragraph}[1]{\oldparagraph{#1}\mbox{}}
\fi
\ifx\subparagraph\undefined\else
\let\oldsubparagraph\subparagraph
\renewcommand{\subparagraph}[1]{\oldsubparagraph{#1}\mbox{}}
\fi

% set default figure placement to htbp
\makeatletter
\def\fps@figure{htbp}
\makeatother

\usepackage{booktabs}
\usepackage{longtable}
\usepackage[bf,singlelinecheck=off]{caption}

%\setmainfont[UprightFeatures={SmallCapsFont=AlegreyaSC-Regular}]{Alegreya}

\usepackage{framed,color}
\definecolor{shadecolor}{RGB}{248,248,248}

\renewcommand{\textfraction}{0.05}
\renewcommand{\topfraction}{0.8}
\renewcommand{\bottomfraction}{0.8}
\renewcommand{\floatpagefraction}{0.75}

\renewenvironment{quote}{\begin{VF}}{\end{VF}}
\let\oldhref\href
\renewcommand{\href}[2]{#2\footnote{\url{#1}}}

\ifxetex
  \usepackage{letltxmacro}
  \setlength{\XeTeXLinkMargin}{1pt}
  \LetLtxMacro\SavedIncludeGraphics\includegraphics
  \def\includegraphics#1#{% #1 catches optional stuff (star/opt. arg.)
    \IncludeGraphicsAux{#1}%
  }%
  \newcommand*{\IncludeGraphicsAux}[2]{%
    \XeTeXLinkBox{%
      \SavedIncludeGraphics#1{#2}%
    }%
  }%
\fi

\makeatletter
\newenvironment{kframe}{%
\medskip{}
\setlength{\fboxsep}{.8em}
 \def\at@end@of@kframe{}%
 \ifinner\ifhmode%
  \def\at@end@of@kframe{\end{minipage}}%
  \begin{minipage}{\columnwidth}%
 \fi\fi%
 \def\FrameCommand##1{\hskip\@totalleftmargin \hskip-\fboxsep
 \colorbox{shadecolor}{##1}\hskip-\fboxsep
     % There is no \\@totalrightmargin, so:
     \hskip-\linewidth \hskip-\@totalleftmargin \hskip\columnwidth}%
 \MakeFramed {\advance\hsize-\width
   \@totalleftmargin\z@ \linewidth\hsize
   \@setminipage}}%
 {\par\unskip\endMakeFramed%
 \at@end@of@kframe}
\makeatother

% \renewenvironment{Shaded}{\begin{kframe}}{\end{kframe}}

\newenvironment{rmdblock}[1]
  {
  \begin{itemize}
  \renewcommand{\labelitemi}{
    \raisebox{-.7\height}[0pt][0pt]{
      {\setkeys{Gin}{width=3em,keepaspectratio}\includegraphics{images/#1}}
    }
  }
  \setlength{\fboxsep}{1em}
  \begin{kframe}
  \item
  }
  {
  \end{kframe}
  \end{itemize}
  }
\newenvironment{rmdnote}
  {\begin{rmdblock}{note}}
  {\end{rmdblock}}
\newenvironment{rmdcaution}
  {\begin{rmdblock}{caution}}
  {\end{rmdblock}}
\newenvironment{rmdimportant}
  {\begin{rmdblock}{important}}
  {\end{rmdblock}}
\newenvironment{rmdtip}
  {\begin{rmdblock}{tip}}
  {\end{rmdblock}}
\newenvironment{rmdwarning}
  {\begin{rmdblock}{warning}}
  {\end{rmdblock}}

\usepackage{makeidx}
\makeindex

\urlstyle{tt}

\usepackage{amsthm}
\makeatletter
\def\thm@space@setup{%
  \thm@preskip=8pt plus 2pt minus 4pt
  \thm@postskip=\thm@preskip
}
\makeatother

\frontmatter

\title{Reading Notes}
\author{Hui Lin}
\date{2018-02-10}

\begin{document}
\maketitle

%\cleardoublepage\newpage\thispagestyle{empty}\null
%\cleardoublepage\newpage\thispagestyle{empty}\null
%\cleardoublepage\newpage
\thispagestyle{empty}
\begin{center}
%\includegraphics{images/dedication.pdf}
\end{center}

\setlength{\abovedisplayskip}{-5pt}
\setlength{\abovedisplayshortskip}{-5pt}

{
\hypersetup{linkcolor=black}
\setcounter{tocdepth}{2}
\tableofcontents
}
\listoftables
\listoffigures
\chapter{Introduction}\label{introduction}

Learning is happening increasingly outside of formal educational
settings and in unsupervised environments. Technological advantages
provide new tools and opportunities for life-time learners. This reading
notes is my first step of managing my learning (aka: Personal Knowledge
Management). This step is called Personal Information Management(PIM).
The goals are:

\begin{enumerate}
\def\labelenumi{\arabic{enumi}.}
\tightlist
\item
  Organize information
\item
  Initially process the information and understand the information
  structure
\item
  Chunking: cluster similar information, filter out some detail, degrade
  the complexity
\end{enumerate}

The reading notes are arranged according to book category in my reading
list since 2009. For some reason I lost all reading information before
2009. There are tons of paper notes need to be digitized and I will keep
working on it in my spare time. My personal comments will start with
``Hui:'' in blue. Sometimes I summarize the contents with my own words.
It is great if you find any of those notes useful for you as well. I
thank all my friends who recommended great books to me. I wrote this
notes with \href{https://rmarkdown.rstudio.com}{R Markdown} and the R
package \href{https://bookdown.org/yihui/bookdown/}{bookdown}.

\mainmatter

\chapter{Cognitive Science}\label{cognitive-science}

\section{Thinking Fast and Slow}\label{thinking-fast-and-slow}

一、大脑的运作机制

1、大脑的运作是双核系统,有快与慢两种系统模式。

系统1:快的思考方式,即直觉系统,就像自动驾驶系统,指人类无意识的快速思考模式,依赖情感、经验和记忆来做决策判断,不怎么费脑力。

系统2:慢的思考方式,就像手动驾驶,指有意识进行的一种更慢、更严谨,需要投入更多脑力的思考模式,不容易出错但它很懒惰,经常走捷径,直接采纳系统1的直觉型判断结果。

2、误区:主导决策和判断的主角是系统2?

其实真正的主角是系统1,系统2只是配角。

因为大多数情况下都是系统1在处理,而系统1用直觉进行思考判断的速度非常快,快到往往在人们还没有意识到就将问题处理完了。而如果让系统2来处理,凡事都通过深思熟虑来做决策,实在是太耗费脑力和精力。

3、系统1和系统2的运作机制

这两个系统之间的互动,构成了大脑认知和思考的运作机制:

大多数情况下是系统1用直觉来快速处理,只有在系统1遇到解决不了的问题时,才求助于系统2,系统2才会调动注意力、分析力来思考和解决问题。

优势:这样既能节约脑力,又能解决绝大多数问题。

弊端:我们容易被系统1也就是直觉所欺骗和糊弄。

原因是这种互动的运作机制存在漏洞,即有时候系统1面对不擅长解决而需要系统2介入的问题,却非要用直觉来做判断和处理,而系统2容易怠倦就盲目相信系统1能处理,最终导致决策失误。

二、欺骗我们的三大直觉漏洞

直觉系统会对我们的观点和行为有无所不在的影响,却存在不可避免的错误和偏颇,导致我们容易被直觉糊弄和欺骗,从而过分自信和决策失误。

直觉漏洞一:简单联想
指做决定并不是深思熟虑根据分析和判断来进行,而是直觉系统启动联想效应,以联想到的相关事情来作为判断的重要依据。

启动效应

也可以说联想效应,指大脑一旦接受一个新的概念或者信息的时候,立刻启动联想这个机制,带动与此相关的其他记忆和感情。

案例一:心理学实验``投票地点对投票结果的影响''

案例二:心理学实验``图像对自助付款的监督效果对比''

案列三:心理学实验``阅读包含老人意象的词组后影响行动力变慢''

案例四:心理学实验``慢速走路后更容易发现类似老人这样的词组''

很多人利用简单联想来影响我们的行为和决策

广告商深谙简单联想的巨大威力,比如可口可乐。

我们自己因简单联想的影响而决策失误

很可能在生活中因为前一次成功的经验就会简单联想到之后也会成功。

比如:赌钱的人,赢钱了就觉得接下来还会赢。

直觉漏洞二:易得性判断
指只利用大脑能够想得起来的信息进行判断。直觉系统的这个漏洞,会给决策带来很多问题。比如:100个因素需要分析,大脑只能想到5个因素,直觉就会只用这5个因素来拼凑一个合理的结论。

故事比统计概率更能说服我们。

直觉系统,宁愿相信貌似合理的结论而不是一个可能性更高的结论。这和逻辑、统计概率相矛盾

案例一:心理学经典案列``琳达实验''

易得性原理

指直觉系统通过易得性的数据信息进行貌似合理的分析判断,而不容易得到的信息,都不会主动去思考判断,因为快思考的模式,而容易忽略重要的关键因素。

比如:因为学生的纹身而判断他成绩不好。因为领导演讲很棒,就觉得他是一个合格的好领导。

如果一件事需要对100个因素进行分析,但是大脑只能想到5个因素,直觉就会只用这5个因素来拼凑一个合理的结论,并且只相信这个结论。

直觉漏洞三:因果联系
指直觉系统习惯于用因果关系解释自己观察到的现象,否认随机性。

对随机性的误解和错觉

未经过统计学训练的人,很难相信随机性,更倾向于用虚假的因果关系去解释他们认为不随机的事件。

案例一:篮球运动中的投篮顺手观点,其实只是随机现象带来的``运气''

在随机事件上寻找因果联系会导致极大的损失

案例二:小样本带来的统计大损失------比尔盖茨基金会赞助17亿美元的无用研究

三、如何避免被直觉的漏洞所欺骗?

解决方案很简单,即在关键时候让系统二介入来处理。
难点在于:质疑直觉系统很难,会让我们有压力,不愉快,想要逃避。这也是为什么直觉系统这些漏洞发生后,我们根本意识不到的原因。

两个实用方法
引入外部意见,并结合核查清单的方法,能帮助我们在关键时刻调动系统二介入,从而避免直觉系统犯错误。

方法一:引入外部意见

指旁观者清,应当多听从外部的信息来做思考判断。

案例一:作者花费8年事件编写出版教材的失败经历

方法二:核查清单

相当于给大脑找一个外部专家,借助清单这样的外部工具,将直觉带来的错觉写进核查清单来提醒自己,在关键时刻抵制直觉系统犯错误。

案例一:查理芒格用误判清单来提醒自己在重大投资和决策的时候不要犯错。

\chapter{Self-development}\label{self-development}

\section{Emotion Management}\label{emotion-management}

\subsection{Emotional First Aid (情绪急救)}\label{emotional-first-aid-}

\begin{quote}
Dr.~Guy Winch is a licensed psychologist, author, and in-demand keynote
speaker whose books have been translated into 24 languages. His first
TED Talk, Why We All Need to Practice Emotional First Aid has been
viewed over 5 million times and is rated as the \#5 most inspiring TED
Talk of all time on ted.com He also writes the popular Squeaky Wheel
Blog on Psychology Today.com
(\url{https://www.psychologytoday.com/blog/the-squeaky-wheel}) and he
has dabbled in stnad-up comedy. His website is www.guywinch.com. You can
view his TED Talks here: \url{https://www.ted.com/speakers/guy_winch}
\end{quote}

Healing Rejection, Guilt, Failure, and Other Everyday Hurts

\subsection{Others}\label{others}

\begin{itemize}
\tightlist
\item
  From TED talk:
  \href{https://www.ted.com/talks/guy_winch_how_to_fix_a_broken_heart?utm_source=newsletter_weekly_2018-02-10\&utm_campaign=newsletter_weekly\&utm_medium=email\&utm_content=talk_of_the_week_image}{How
  to fix a broken heart}
\end{itemize}

Heartbreak is far more insidious than we realize. There is a reason we
keep going down one rabbit hole after another, even when we know it's
going to make us feel worse. Brain studies have shown that the
withdrawal of romantic love activates the same mechanisms in our brain
that get activated when addicts are withdrawing from substances like
cocaine or opioids.

Getting over heartbreak is not a journey. It is a fight, and your reason
is your strongest weapon. There is no breakup explanation that is going
to feel satisfying. No rationale can take away the pain you feel. So
don't search for one. Don't wait for one, just accept the one you were
offered or make up one yourself and then put the question to rest,
because you need that closure to resist the addition. And you need
something else as well: you have to be willing to let go, to accept that
it's over. Otherwise, your mind will feed on your hope and set you back.
Hope can be incredibly destructive when your heart is broken.

\bibliography{bibliography.bib}

\backmatter
\printindex

\end{document}
